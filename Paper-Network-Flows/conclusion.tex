\section{Conclusions}\label{sec:conclusions}

In this paper, we propose a method for detecting Android malware by combining the \mas with \net. We use the same dataset (\cds) as the \fhc to investigate whether our method can overcome the limitations of the \mas, point by the \fhc, particularly in families that heavily interact with networks. Our evaluation demonstrates that by combining the \mas with \net, we achieve good performance in detecting mobile malware, with an \fone of $0.91$. Although \net leverage the MAS approach's results, we show that it is not a complete solution. Our evaluation reveals that there are samples in the \cds which \net fault in flag as malicious, but the \mas do not, indicating that both approaches complement each other. Our paper also highlights the limitations posed by the malicious samples, discussing the importance of the number of malware samples used for training, as this can affect the accuracy of machine learning algorithms and, consequently, the support of Network Traffic Analysis for the \mas. As future work, we plan to collect, train, and analyze more malware samples to improve the malware detection solution by developing more sophisticated models, that could provide better support for the \mas.