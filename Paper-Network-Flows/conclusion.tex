\section{Conclusions}\label{sec:conclusions}



In this paper, we propose a framework (\droidxpflow) for detecting Android malware using Network Traffic Analysis with ML support. As a first step, we created a dataset of network traffic (\fds) from the execution of $4,067$ repackaged apps, extracted from the dataset presented in the previous chapter. We then used our framework to investigate whether our method could overcome the limitations of the \mas, as discussed in the previous chapter, particularly for malware families that heavily interact with networks. Our evaluation demonstrates that \droidxpflow achieves a good performance in detecting Android malware, with an \fone of $0.89$. Although \droidxpflow builds upon the results of the state-of-the-art Mining Sandbox, we show that it is not a complete solution. Our evaluation reveals that there are samples in \fds that our framework fails to flag as malicious, whereas the \mas succeeds. We also highlight the limitations posed by our malicious samples quantity and discuss the importance of the number of samples used for training, as this can affect the accuracy of ML algorithms and, consequently, the effectiveness of our approach. As future work, we plan to collect, train, and analyze more malware samples to improve our malware detection solution by developing more sophisticated models. Additionally, we intend to explore more recent test generation tools that can better simulate user input, thereby making the collected traffic more closely resemble real-world scenarios.
