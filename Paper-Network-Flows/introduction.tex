\section{Introduction}\label{sec:introduction}

Mobile technologies like smartphones and tablets have become fundamental to the way we function as a society. Almost two-thirds of the world population uses mobile technologies~\cite{Comscore,DBLP:journals/tse/MartinSJZH17}, with the
Android Platform dominating this market and accounting for more than 70\% of the \emph{mobile market share}, with almost 2.5 million Android applications~\footnote{In this paper, we will use the terms Android Applications, Android Apps, and Apps interchangeably, to refer to Android software applications} (apps) available on the Google Play Store, in June 2023~\cite{Statista}.  
As popularity rises, so does the risk of potential attacks, prompting collaborative efforts from both academia and industry to design and develop new techniques for identifying malicious behavior or vulnerable code in Android apps~\cite{10.1145/3017427}.

%added by Roberto
A prevalent form of Android malware involves repackaging legitimate apps~\cite{DBLP:conf/wcre/BaoLL18, le2018towards}. These malicious variants embed harmful code, such as mechanism to leak sensitive data to external servers~\cite{DBLP:journals/tse/LiBK21} and are often distributed through official app stores. To counter this, the \mas was developed to create sandboxes by analyzing sensitive API calls~\cite{DBLP:conf/icse/JamrozikSZ16}. The method consists of two stages; first, automated tools examine the app, identifying interactions with sensitive APIs, subsequently, during normal app execution, the sandbox blocks any API calls not previously observed, effectively isolating potential malicious behavior (Figure~\ref{fig:mine}). 

Initially focused on app behavior abstraction, it has proven effective in detecting repackaged malware, as reported at previous work~\cite{DBLP:conf/wcre/BaoLL18}, which reported a accuracy of $75.5\%$ (77 out of 102 app pairs~\footnote{When we use the term App pair, we refer to the pairs (original and repackaged Android Application)}). However, Bao et al. study~\cite{DBLP:conf/wcre/BaoLL18} only evaluated the technique using 102 app pairs, with a restrict number of malware families. 

%Here we can add my work ...
%The lack of understanding about Mining Android Sandbox also appears
%in the work of Francisco et al. (hereafter FHC-Study), which presents an empirical study that explores the
%performance of ...  for identifying malicious behavior using the mining sandbox approach.

Also, in response to the escalating mobile malware threat, innovative network-based detection's methods are explored. This approach holds promise as malicious activities often manifest through network connections. In this study, we focus on comprehensively analyzing mobile network traffic to identity mobile malware. We examine the traffic generated by both malicious and benign apps collected with \droidxp using TcpDump tool. Then, feature engineering is done on the obtained dataset, preparing it to unsupervised machine learning techniques. Finally both \mas and Flow Analysis are combined exploring how they complement each other to identify malware on Android apps.

The rest or the paper is organized as follows...

\begin{small}
  \begin{center}
    \url{https://github.com/droidxp/ML}
  \end{center}
\end{small}

