\section{Introduction}\label{sec:introduction}

Mobile technologies like smartphones and tablets have become fundamental to the way we function as a society. Almost two-thirds of the world population uses mobile technologies~\cite{Comscore,DBLP:journals/tse/MartinSJZH17}, with the
Android Platform dominating this market and accounting for more than 70\% of the \emph{mobile market share}, with almost 2.5 million Android applications~\footnote{In this paper, we will use the terms Android Applications, Android Apps, and Apps interchangeably, to refer to Android software applications} (apps) available on the Google Play Store, in June 2023~\cite{Statista}.  
As popularity rises, so does the risk of potential attacks, prompting collaborative efforts from both academia and industry to design and develop new techniques for identifying malicious behavior or vulnerable code in Android apps~\cite{10.1145/3017427}.

%added by Roberto
A prevalent form of Android malware involves repackaging legitimate apps~\cite{DBLP:conf/wcre/BaoLL18, le2018towards}. These malicious variants embed harmful code, such as mechanism to leak sensitive data to external servers~\cite{DBLP:journals/tse/LiBK21} and are often distributed through official app stores. To counter this, the Mining Android Sandbox (hereafter \mas) was developed to create sandboxes by analyzing sensitive API calls~\cite{DBLP:conf/icse/JamrozikSZ16}. The method consists of two stages; first, automated tools examine the app, identifying interactions with sensitive APIs, subsequently, during normal app execution, the sandbox blocks any API calls not previously observed, effectively isolating potential malicious behavior (Figure~\ref{fig:mine}).

Also, in response to the escalating mobile malware threat, innovative network-based detection's methods are explored. This approach holds promise, as malicious activities often reveal themselves through network communication~\cite{DBLP:conf/sp/ZhouJ12}. In most cases, malware executes its malicious activities by communicating with a remote server, leaving traces that can easily expose its presence~\cite{DBLP:journals/jnca/YangHHM18}


Focused on app behavior abstraction, the MAS approach has proven effective in detecting repackaged malware, as demonstrated in previous work~\cite{DBLP:conf/wcre/BaoLL18},  which reported an accuracy of $75.5\%$ (77 out of 102 app pairs~\footnote{When we use the term App pair, we refer to the pairs (original and repackaged Android Application)}). However, the study by Bao et al.~\cite{DBLP:conf/wcre/BaoLL18} (hereafter \blls), evaluated the technique using only 102 app pairs, with a limited number of malware families. The need for further exploration of the MAS approach was also identified in the work of F. Costa et al. (hereafter referred to as the FHC-Study), which recognized the necessity for additional studies beyond the \blls. 

Thus, the \fhc presents an empirical evaluation of the \mas on a larger dataset (hereafter \cds), which is significantly larger than the one used in the \blls, containing $4,076$ app pair, and $116$ malware families. The \fhc reveals when considering the \cds, the accuracy of the \mas drops significantly, with an F1-score of $0.54$, compared to the results reported previously. This suggests that the applicability of the \mas for detecting and preventing malicious behaviors cannot be generalized to larger datasets.


Motivated by the limited results reported in the \fhc, this study focuses on comprehensively analyzing mobile network traffic to identify malicious apps. We examine the traffic generated by both malicious and benign apps, utilizing the \droidxp~\cite{DBLP:conf/scam/CostaMCMVBC20}. Under the hood, DroidXP leverages the TcpDump tool to collect traffic data generated by malware and benign apps, respectively. Feature engineering is then performed on the collected dataset to prepare it for unsupervised machine learning techniques.

Finally, both the \mas and \net are combined, exploring how they complement each other in identifying malware in Android apps. Our experimental results reveal a significantly accuracy (\fone of $0.91$) for the \mas combined with \net, compared to what has been reported in the \fhc and \blls, especially in malware families that showed low false negative rates in those studies.\newline\newline
\textbf{Organization.} The rest or the paper is organized as follows: Section~\ref{sec:background} highlights the background and related work. Section~\ref{sec:Methodology} discuss the methodology in details. The results of your approach are discussion in Section~\ref{sec:results}. After present implications and limitations at Section~\ref{sec:discussion}, we close with conclusion at Section~\ref{sec:conclusions}.


