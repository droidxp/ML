\section{Introduction}\label{sec:introduction}


Android is a robust Linux-based operating system widely used in mobile technology. It has more than $2.5$ million Android applications~\footnote{In this paper, the terms Android Applications, Android Apps, and Apps will be used interchangeably to refer to software applications for the Android platform.} (apps) available in the official Google Play Store until June 2023~\cite{Statista}. As its popularity rises, so does the risk of potential attacks, making Android-based devices prime targets for malicious apps (malware). In general, the main aim of malware is to gain unauthorized access to and exploit sensitive resources on a device~\cite{DBLP:conf/ccs/FeltFCHW11,DBLP:journals/eswa/SurendranTE20}.
This can lead to various risks, including disrupted device functionality, battery drain, information leakage, and other threats~\cite{DBLP:conf/ccs/FeltFCHW11,DBLP:conf/sp/ZhouJ12}.

A prevalent form of Android malware involves repackaging legitimate apps~\cite{DBLP:conf/wcre/BaoLL18, le2018towards}. These malicious variants can insert or modify the original apps with harmful code and release them on (un)official third-party markets~\cite{DBLP:journals/tdsc/TianYRTP20}. Researchers~\cite{DBLP:journals/tdsc/TianYRTP20,DBLP:conf/sp/ZhouJ12,DBLP:journals/compsec/MerloRSV21} show that the $86\%$ of Android malicious apps are repackaged, highlighting theprevalence of this approach to inject malicious behavior. To counter this, several general-purpose Android malware detection techniques have been developed. For example, the Mining Android Sandbox (hereafter \mas) for malware detection, adapted from~\cite{DBLP:conf/icse/JamrozikSZ16}, relies on the calls to sensitive APIs to check whether a repackaged version of an app is malicious or not~\cite{DBLP:conf/wcre/BaoLL18,DBLP:jourals/jjc/Handrick22}. The original \mas leverages static and dynamic analysis on Android app to protect sensitive resources at a fine-grained level by limiting access to sensitive APIs.

Focused on app behavior abstraction, the \mas has proven effective in detecting repackaged malware, as demonstrated in previous work~\cite{DBLP:conf/wcre/BaoLL18}, which classified as malware 77 out of 102 app pairs (original and repackaged versions of an app)~\footnote{Hereafter, when we use the term app pair(s), we refer to original and repackaged versions of an Android application}. However, the study by Bao et al.~\cite{DBLP:conf/wcre/BaoLL18} (\blls), evaluated the technique using a dataset comprising only 102 app pairs, with a limited number of malware families. Using the same dataset from \blls, Costa et al.~\cite{DBLP:jourals/jjc/Handrick22}, present an in-depth analysis of \mas highlighting the contributions of the static and dynamic analysis components to malware detection, bringing evidence that both techniques complement each other. 

Further exploration of the \mas was also discussed \pw, which revealed the need for additional studies using datasets larger than used in the \blls. The research presents an empirical evaluation of the \mas using a larger dataset (hereafter referred to as \cds), which contains $4,076$ app pairs and $116$ malware families. That previous study also showed evidence that, when applied to the \cds, the accuracy of the \mas drops significantly, with an \fone of $0.54$. This suggests that the effectiveness of the \mas in detecting and preventing malicious behaviors may not be generalizable to larger datasets. 

Motivated by the negative results reported in our previous research, in this \chap, we (a) leverage our
\droidxp infrastrucutre~\cite{DBLP:conf/scam/CostaMCMVBC20} to collect the network traffic of the apps (while they execute using a
test generation tool like DroidBot~\cite{DBLP:conf/icse/LiYGC17}) and (b) explore machine learning (ML) algorithms to classify
the repackaged version of the apps as malware / non-malware using network traffic data. The results
of the experiments that we present here show that combining the \mas with the ML technique we
detail in this paper leads to an accuracy (\fone) of 0.89. This improvement is particularly significant for malware
families that previously exhibited high false negative rates in our earlier study. Altogether, the main
contributions of this paper are:

\begin{itemize}
  \item {\bf \droidxpflow:} a novel dynamic analysis approach for Android malware detection that relies on network traffic data
  collected using the first phase of the \mas and ML algorithms.

  \item {\bf An empirical study:} that brings evidence that \droidxpflow outpeforms the \mas. In particular, \droidxpflow
  is able to correctly classify the malware families (such as Gappusin) the original version of the
  \mas approach was not. 
\end{itemize}

The main implication of this research is that we shed light to
a possible limitation of the \mas in general. We argue that
robust sandboxes should consider not only the calls to
sensitive APIs, but also monitor network traffic data and the
access to native resources to detect possible malicious
behavior of Android apps.

% \textbf{Our contribution:} Altogether, the main contribution of this paper is two-fold: first, we propose \droidxpflow,
% a framework designed to detect malicious apps based on dynamic analysis of network traffic and ML algorithms.
% To achieve this, the framework collects traffic generated by both original and repackaged versions of the apps using an extension
% of \droidxp~\cite{DBLP:conf/scam/CostaMCMVBC20}. In addition to gathering data related to calls to sensitive APIs, the \droidxp extension also captures network traffic data from the apps using the TcpDump tool. Feature engineering is then applied to extract and select relevant features for training, and characterizing network flows as either benign or malicious using supervised ML algorithms. Second, we generate a labeled and balanced dataset called \fds of benign and malicious flows using the CICFlowMeter~\cite{DBLP:conf/icissp/LashkariDMG17} software. This dataset contains more than $3,000$ network traffic features, extracted from $2,958$ benign apps and $2,886$ malicious apps spanning 116 malware families.\newline\newline

% \textbf{Organization.} The rest of the paper is organized as follows: Section~\ref{sec:background} highlights the background and related work. Section~\ref{sec:Methodology} discuss the studies setting in details. The results of your approach are discussion in Section~\ref{sec:results}. After present implications and limitations at Section~\ref{sec:discussion}, we close with conclusion at Section~\ref{sec:conclusions}.


