\section{Introduction}\label{sec:introduction}

Android is a powerful operating system based on Linux, commonly used in mobile technologies. It has more than $2.5$ million Android applications~\footnote{In this paper, we will use the terms Android Applications, Android Apps, and Apps interchangeably, to refer to Android software applications} (apps) available in the official Google Play Store until June 2023~\cite{Statista}. As its popularity rises, so does the risk of potential attacks, making Android-based devices prime targets for malicious apps (malware). In general, the main aim of malware is to gain unauthorized access to and exploit sensitive resources on a device~\cite{DBLP:conf/ccs/FeltFCHW11,DBLP:journals/eswa/SurendranTE20}. This can result in several issues, such as disruption of the device's normal functioning, battery drainage, information leakage, and more~\cite{DBLP:conf/ccs/FeltFCHW11,DBLP:conf/sp/ZhouJ12}.

A prevalent form of Android malware involves repackaging legitimate apps~\cite{DBLP:conf/wcre/BaoLL18, le2018towards}. These malicious variants can insert or modify the original apps with harmful code and release them on unofficial third-party markets as new~\cite{DBLP:journals/tdsc/TianYRTP20}. Researchers~\cite{DBLP:journals/tdsc/TianYRTP20,DBLP:conf/sp/ZhouJ12} show that more than $80\%$ of malware are repackaged, highlighting the significant prevalence of these malicious apps currently. To counter this, several general-purpose Android malware detection techniques have been developed. For example, the Mining Android Sandbox (hereafter \mas) was created to analyze sensitive API calls by using sandboxes~\cite{DBLP:conf/icse/JamrozikSZ16}, as described at Figure~\ref{fig:mine}.

Focused on app behavior abstraction, the \mas has proven effective in detecting repackaged malware, as demonstrated in previous work~\cite{DBLP:conf/wcre/BaoLL18},  which reported an accuracy of $75.5\%$ (77 out of 102 app pairs~\footnote{When we use the term App pair, we refer to the pairs (original and repackaged Android Application)}). However, the study by Bao et al.~\cite{DBLP:conf/wcre/BaoLL18} (hereafter \blls), evaluated the technique using only 102 app pairs, with a limited number of malware families. The need for further exploration of the MAS approach was also identified in the work of F. Costa et al. (hereafter referred to as the \fhc), which recognized the necessity for additional studies beyond the \blls.

Thus, the \fhc presents an empirical evaluation of the \mas on a larger dataset (hereafter \cds), which is significantly larger than the one used in the \blls, containing $4,076$ app pair, and $116$ malware families. The \fhc reveals when considering the \cds, the accuracy of the \mas drops significantly, with an F1-score of $0.54$, compared to the results reported previously. This suggests that the applicability of the \mas for detecting and preventing malicious behaviors cannot be generalized to larger datasets.


Motivated by the limited results reported in the \fhc, this study focuses on comprehensively analyzing mobile network traffic to identify malicious apps. We examine the traffic generated by both malicious and benign apps, utilizing the \droidxp~\cite{DBLP:conf/scam/CostaMCMVBC20}. Under the hood, DroidXP leverages the TcpDump tool to collect traffic data generated by malware and benign apps, respectively. Feature engineering is then performed on the collected dataset to prepare it for unsupervised machine learning techniques.

Finally, both the \mas and \net are combined, exploring how they complement each other in identifying repackaged malware in Android apps. Our experimental results reveal a significantly accuracy (\fone of $0.91$) for the \mas combined with \net, compared to what has been reported in the \fhc and \blls, especially in malware families that showed low false negative rates in those studies.\newline\newline
\textbf{Organization.} The rest of the paper is organized as follows: Section~\ref{sec:background} highlights the background and related work. Section~\ref{sec:Methodology} discuss the methodology in details. The results of your approach are discussion in Section~\ref{sec:results}. After present implications and limitations at Section~\ref{sec:discussion}, we close with conclusion at Section~\ref{sec:conclusions}.


