\documentclass[10pt,journal,compsoc]{IEEEtran}

\pdfminorversion=7

\usepackage{amsmath,amsfonts}
\usepackage{algorithmic}
\usepackage{array}
\usepackage[caption=false,font=normalsize,labelfont=sf,textfont=sf]{subfig}
\usepackage{textcomp}
\usepackage{stfloats}
\usepackage{url}
\usepackage{verbatim}
\usepackage{graphicx}
\usepackage{booktabs}
\usepackage{enumerate}
\usepackage{listings}
\usepackage{hyperref}
\usepackage{xspace}
\usepackage{balance}
\usepackage[most]{tcolorbox}
\usepackage{awesomebox}
\usepackage{multirow}
\usepackage{blindtext}

\usepackage{mdframed}
\usepackage[colorinlistoftodos]{todonotes}

\hyphenation{op-tical net-works semi-conduc-tor IEEE-Xplore}
\def\BibTeX{{\rm B\kern-.05em{\sc i\kern-.025em b}\kern-.08em
    T\kern-.1667em\lower.7ex\hbox{E}\kern-.15emX}}

%.1667


% Custom colors
\definecolor{keywords}{rgb}{0.5,0,0.35}
\definecolor{comments}{RGB}{0,0,113}
\definecolor{red}{RGB}{160,0,0}
\definecolor{green}{RGB}{0,150,0}
 
\lstset{language=Python, 
        basicstyle=\ttfamily\small, 
        keywordstyle=\color{keywords}\bfseries,
        commentstyle=\color{comments},
        stringstyle=\color{red},
        showstringspaces=false,
        %identifierstyle=\color{red},
       }
       
\newmdtheoremenv [
 outerlinewidth = 1 ,
 roundcorner = 1pt,
 leftmargin = 1,
 rightmargin = 1,
 backgroundcolor = gray!20,
 outerlinecolor = blue!70!black,
 %innertopmargin = \topskip, 
 %splittopskip = \topskip ,
 ntheorem = false,
] {finding}{Finding}

% Pictures
\usepackage{graphicx}       



\newtcbtheorem{obs}{Finding}{%
        theorem name,%
        colback=gray!5,%
        colframe=gray!35!black,%
        fonttitle=\bfseries,title after break={Lemma  -- \raggedleft Continued}%
    }{lem}


% \newcommand{\pw}{Chapter~\ref{chap:large-study}}    
% \newcommand{\pw}{\ref{}}    
\newcommand{\pw}{in Chapter~5}

% \newcommand{\chap}{section}
\newcommand{\chap}{chapter}

\newcommand{\tb}[2]{\tipbox{{\bf Finding #1}. #2}}

\newcommand{\droidxp}{DroidXP\xspace}
\newcommand{\droidxpflow}{DroidXPflow\xspace}
\newcommand{\review}[1]{\textcolor{black}{#1}}
\newcommand{\alert}[1]{\textcolor{red}{#1}}
\newcommand\kn[1]{\textcolor{red}{KN: #1}}
\newcommand\fh[1]{\textcolor{green}{FH: #1}}
\newcommand\rb[1]{(\textcolor{red}{RB: #1})}

\newcommand{\highlight}[1]{{\color{red}}#1}

\newcommand\raw[1]{\textcolor{red}{#1}\xspace}

\newcommand{\mas}{MAS approach\xspace}

\newcommand{\net}{Network Traffic Analysis\xspace}

\newcommand{\ml}{ML process\xspace}

\newcommand{\fm}[1]{\emph{#1}\xspace}

\newcommand{\gps}{\fm{gappusin}}  % the gappusin family.
\newcommand{\rmb}{\fm{revmob}}
\newcommand{\dwg}{\fm{dowgin}}

\newcommand{\tjk}{\fm{torjok}}

\newcommand{\sscore}{Similarity Score\xspace}

\newcommand{\rqa}{How much gain we obtain on the accuracy of Android malware classification when considering network flow data and ML?}

\newcommand{\rqb}{Which algorithms are effective to be used for training models base on network traffic data and ML for malware identification?}

\newcommand{\rqc}{Which  malware families (e.g., \fm{gappusin}, \fm{kuguo}, \fm{dowgin}) the Flow Analysis has the best performance, and in which families the proposal is not so efficient?}

\newcommand{\rqe}{How much gain we obtain on the performance of the \mas for malware classification when considering its extensions?}


\newcommand{\repack}{RePack\xspace}
\newcommand{\amc}{AndroMalPack\xspace}

\newcommand{\appsSmall}{102\xspace}
\newcommand{\apps}{\textcolor{black}{4,076}\xspace}
\newcommand{\napps}{\textcolor{blue}{726}\xspace}

\newcommand{\sds}{\texttt{SmallDS}\xspace}
\newcommand{\cds}{\texttt{LargeDS}\xspace}
\newcommand{\fds}{\texttt{FlowDS}\xspace}
\newcommand{\nds}{\texttt{DS3}\xspace}
\newcommand{\avt}{\texttt{avclass2} tool\xspace}
\newcommand{\vt}{\texttt{VirusTotal}\xspace}
\newcommand{\se}{security engine\xspace}
\newcommand{\ses}{security engines\xspace}


\newcommand{\fone}{F1-score\xspace}
\newcommand{\fscoreSmall}{0.89\xspace}
\newcommand{\fscoreNew}{0.85\xspace}

\newcommand{\nfscoreSmall}{0.85\xspace}
\newcommand{\nfscoreSmallC}{0.87\xspace}

\newcommand{\fscore}{\textcolor{black}{0.54}\xspace}
\newcommand{\fscoreC}{\textcolor{blue}{0.49}\xspace}

\newcommand{\malwares}{\textcolor{black}{2,895}\xspace}
\newcommand{\malwaresP}{\textcolor{black}{71.02}}
\newcommand{\malwaresN}{\textcolor{blue}{87.98}}
\newcommand{\appsGps}{\textcolor{black}{1,337}\xspace}
\newcommand{\appsGpsFN}{\textcolor{black}{1,170}\xspace}
\newcommand{\fhc}{FHC-Study\xspace}
\newcommand{\blls}{BLL-Study\xspace}


\begin{document}

\title{Using Network Flow Data and Machine Learning as Support for Mining Android Sandbox}


\author{Francisco Costa,
        Roberto Valera, 
        Rodrigo~Bonif\'{a}cio,
        Eduardo Gomes, 
        Jo\~{a}o Gondim
\IEEEcompsocitemizethanks{
\IEEEcompsocthanksitem F. Costa, R. Valera, R. Bonif\'{a}cio, E. Gomes and J. Gondim are with the 
Computer Science Department, University of Bras\'{i}lia, Bras\'{i}lia, Brazil.
E-mail: \{francisco.costa\}@aluno.unb.br and \{roberto.luis,rbonifacio,edumonteiro,gondim\}@unb.br.

}
}



\IEEEtitleabstractindextext{
\begin{abstract}
As mobile technologies become part of modern society, Android's dominance in the global smartphone market has made it a prime target for cyberattacks. The Android platform faces a growing threat from malware, particularly repackaged apps that embed malicious code to exploit user data. While sandbox-based approaches, such as Mining Android Sandbox (MAS), have been effective in detecting repackaged malware by analyzing sensitive API interactions, they often fall short in identifying more complex threats. This paper introduces a complementary approach that integrates network flow analysis with \mas to improve malware detection. By analyzing the network traffic generated by Android applications, alongside behavioral data from \mas, this method uncovers malicious activities that may remain hidden within the app’s behavior. Using machine learning on network traffic data collected via TcpDump, combined with \mas insights, our method improves the accuracy of malware detection. Results demonstrate that this combined approach significantly improves the identification of malware, offering a more comprehensive solution for securing Android applications against evolving threats.
\end{abstract}
}





\maketitle
\begin{IEEEkeywords}
Android Malware Detection, Dynamic Analysis, Network Traffic Analysis, Smartphone, Mining Android Sandboxes.
\end{IEEEkeywords}

\section{Introduction}\label{sec:introduction}

Android is a powerful operating system based on Linux, commonly used in mobile technologies. It has more than $2.5$ million Android applications~\footnote{In this paper, we will use the terms Android Applications, Android Apps, and Apps interchangeably, to refer to Android software applications} (apps) available in the official Google Play Store until June 2023~\cite{Statista}. As its popularity rises, so does the risk of potential attacks, making Android-based devices prime targets for malicious apps (malware). In general, the main aim of malware is to gain unauthorized access to and exploit sensitive resources on a device~\cite{DBLP:conf/ccs/FeltFCHW11,DBLP:journals/eswa/SurendranTE20}. This can result in several issues, such as disruption of the device's normal functioning, battery drainage, information leakage, and more~\cite{DBLP:conf/ccs/FeltFCHW11,DBLP:conf/sp/ZhouJ12}.

A prevalent form of Android malware involves repackaging legitimate apps~\cite{DBLP:conf/wcre/BaoLL18, le2018towards}. These malicious variants can insert or modify the original apps with harmful code and release them on unofficial third-party markets~\cite{DBLP:journals/tdsc/TianYRTP20}. Researchers~\cite{DBLP:journals/tdsc/TianYRTP20,DBLP:conf/sp/ZhouJ12,DBLP:journals/compsec/MerloRSV21} show that the $86\%$ of malware are repackaged, highlighting the significant prevalence of these malicious apps currently. To counter this, several general-purpose Android malware detection techniques have been developed. For example, the Mining Android Sandbox (hereafter \mas) was created to analyze sensitive API calls by using sandboxes~\cite{DBLP:conf/icse/JamrozikSZ16}. The approach performs static and dynamic analysis on Android app to protect sensitive resources at a fine-grained level by limiting access to sensitive APIs.

Focused on app behavior abstraction, the \mas has proven effective in detecting repackaged malware, as demonstrated in previous work~\cite{DBLP:conf/wcre/BaoLL18}, which reported an accuracy of $75.5\%$ (77 out of 102 app pairs~\footnote{When we use the term App pair, we refer to the pairs (original and repackaged Android Application)}). However, the study by Bao et al.~\cite{DBLP:conf/wcre/BaoLL18} (hereafter \blls), evaluated the technique using only 102 app pairs, with a limited number of malware families, and do not explored the role of static and dynamic analysis on malicious behavior detection. Using the same app samples from \blls, Costa et al.~\cite{DBLP:jourals/jjc/Handrick22}, present an in-depth analysis related to the importance of static and dynamic analysis at malware detection, bringing evidence that both techniques complement each other, on Android malware classification task. 

Further exploration of the \mas was also identified at chapter 5 of this thesis, which recognized the necessity for additional studies beyond the \blls. The research presents an empirical evaluation of the \mas using a larger dataset (hereafter referred to as LargeDS), which is significantly bigger than the one used in the \blls. This dataset contains $4,076$ app pairs and $116$ malware families. The study reveals that when applied to the \cds, the accuracy of the \mas drops significantly, with an \fone of $0.54$, compared to the previously reported. This suggests that the effectiveness of the \mas in detecting and preventing malicious behaviors may not be generalizable to larger datasets. 

Motivated by the limited results reported in the \blls, this research focuses on the dynamic behavior of malware and aims to detect it by analyzing its network traffic features, with the support of machine learning (ML) algorithms. Our experimental results show that the proposed model outperforms previous state-of-the-art approaches, achieving an \fone of 0.89. This improvement is particularly significant for malware families that previously exhibited high false negative rates in earlier studies.\newline
\textbf{Our contribution:} The main contribution of this paper is two-fold: first, we propose \droidxpflow, a framework designed to detect malicious apps based on dynamic analysis of network traffic and decision-making. To achieve this, the framework collects traffic generated by both malicious and benign apps using an extension of \droidxp~\cite{DBLP:conf/scam/CostaMCMVBC20}. In addition to gathering data related to calls to sensitive APIs, the \droidxp extension also captures network traffic data from the apps using the TcpDump tool. Feature engineering is then applied to extract and select relevant features for training, and characterizing network flows as either benign or malicious using supervised ML algorithms. Second, we generate a labeled and balanced dataset called \fds of benign and malicious flows using the CICFlowMeter~\cite{DBLP:conf/icissp/LashkariDMG17} software. This dataset contains more than $3,000$ network traffic features, extracted from $2,958$ benign apps and $2,886$ malicious apps spanning 116 malware families.\newline\newline

\textbf{Organization.} The rest of the paper is organized as follows: Section~\ref{sec:background} highlights the background and related work. Section~\ref{sec:Methodology} discuss the studies setting in details. The results of your approach are discussion in Section~\ref{sec:results}. After present implications and limitations at Section~\ref{sec:discussion}, we close with conclusion at Section~\ref{sec:conclusions}.



\section{Background and Related Work}\label{sec:background}

Extensive research has focused on pre-installation Android malware detection through static and dynamic analysis. Dynamic approaches, like TaintDroid~\cite{DBLP:conf/osdi/EnckGCCJMS10}, DroidRanger~\cite{Zhou2012HeyYG}, and DroidScope~\cite{LKYanDroidscope}, monitor app behavior in real-time, offering high accuracy but significant performance overhead, limiting their practical use on mobile devices. In contrast, static methods such as Kirin~\cite{Enck2009}, Stowaway~\cite{DBLP:conf/ccs/FeltCHSW11}, and RiskRanker~\cite{GraceRiskranker2012} are efficient and scalable but heavily rely on manually defined patterns, hindering their effectiveness against novel malware. Additionally, these methods often lack transparency, making it difficult to understand their decision-making processes. The lack of understanding about Mining Android Sandbox (\mas) also appears in the \fhc, which presents an empirical study that explores the performance of \mas when using a large dataset of pair of apps for identifying malicious behavior.

%\subsection{Mining Android Sandbox}

%A sandbox is a controlled environment that isolates applications from the host system, preventing them from accessing or modifying files, networks, or other device data ~\cite{DBLP:journals/peerj-cs/MaassSCS16}. This isolation allows for safe testing and execution of potentially malicious code without compromising the device's integrity~\cite{DBLP:conf/esorics/BordoniCS17}. Such a need arises in various scenarios, including when dealing with untrusted user input, analyzing malware, or mitigating risks in compromised systems~\cite{DBLP:journals/peerj-cs/MaassSCS16}. A sandbox must protect the host machine and operating system from any harm caused by third-party software. To achieve this, it should provide the minimum necessary resources for program execution, ensuring that the program does not impact external resources.

%The \mas ~\cite{DBLP:conf/icse/JamrozikSZ16} employs test generation tools to examine an Android app's dynamic behavior and identify essential sensitive resources. By restricting access to these specific APIs, the sandbox safeguards app execution. The process involves two stages. In the exploratory phase, a benign app version is executed using test generation tools, recording the utilized sensitive APIs. Subsequently, during the execution phase, the sandbox limits the app's access to only the previously identified sensitive APIs, preventing malicious apps from accessing any unauthorized sensitive resources.

%Beyond its ability to generate Android sandboxes, the MAS approach is also effective in identifying malicious behavior in repackaged Android apps~\cite{DBLP:conf/wcre/BaoLL18}. The effectiveness of this approach is measured by its accuracy in correctly flagging malicious activities within repackaged versions of applications. In ~\cite{DBLP:jourals/jjc/Handrick22} the authors explore the use of static and dynamic analysis to improve the performance of the \mas. They propose a new approach based on taint analysis for malware identification demonstrating that when combining taint analysis with \mas, the percentage of malware identification is increased. Despite this results, both \mas and taint analysis present limitations that can be overcome using other approaches like, machine learning.

\subsection{Network Traffic Analysis}

System calls and network traffic are two key behaviors commonly analyzed in the dynamic analysis of Android apps. Recently, increased attention has been directed toward the network traffic generated by malware. As a result, researchers have begun analyzing and identifying malicious apps based on their network traffic. For example, signature-based detection methods assess malware by comparing it to known malware patterns. Griffin et al.~\cite{Griffin2009}  used 48-byte code sequences as signatures. Researchers have also explored automatically generating network signatures~\cite{PolygraphNewsome2005, Singh2004, Yegneswaran2005}, often focusing on worm identification. Perdisci et al. ~\cite{Perdisci2010} generated network signatures for mobile malware based on HTTP traffic, analyzing similarities and clustering malicious patterns. While effective against known threats, signature-based methods struggle to detect novel attacks due to their reliance on predefined patterns.

Some studies utilize text analysis for malware detection based on Packet/Flow textual features. Nan et al.~\cite{yuhong:usenix-2015} introduced UIPicker, a framework that uses NLP, machine learning, and program analysis to identify personal user information on a large scale. N-grams, a technique from NLP, have been applied to network protocol identification~\cite{YunWang2016}. Recon et al. ~\cite{ren:mobisys-2016} recently proposed a method to detect and prevent personal information leaks in mobile network traffic by analyzing key-value pairs. Some authors has also used Packet/Flow features statistically. Arora et al.~\cite{arora:ngmast-2014} compared malware traffic to benign network traffic, identifying deviations in network behavior using statistical features like average packet size, flow duration, and byte ratios. AppScanner~\cite{taylor:eurosp-2016} is a framework that uses statistical features of encrypted network traffic to automatically fingerprint and identify Android apps. Conti et al.~\cite{Conti2016} [19] analyzed encrypted Android traffic to identify user actions based on statistical features. However, statistical feature-based methods can have a high error rate due to their coarse-grained traffic characterization.

\subsection{Machine Learning (ML) Approach}

ML approach has been used in network traffic-based malware detection methods. Depending on the type of ML model used, these methods can be divided into two categories: (1) shallow learning techniques and (2) deep learning techniques. Shallow learning usually relies on handcrafted features based upon the target problem. These techniques include classic ML methods such as decision tree, random forest, KNN, and SVM. In contrast, deep learning methods are able to derive their own features directly from data by different hidden neural network layers.

Shallow techniques traditionally are used in malware detection methods as a classification problem. Researchers~\cite{RIBEIRO2020} have developed a host-based application to monitor device resource usage (e.g., CPU, memory, battery) and detect malware with over $90$\% accuracy using statistical and ML methods. Chen et al.~\cite{CHEN2018346} investigated the impact of data imbalance on Android malicious flow detection, finding that it can significantly reduce accuracy. They experimented with various classifiers on imbalanced datasets, demonstrating the effectiveness of ML for identifying malicious mobile traffic. Another study~\cite{lashkari:pst-2017} analyzed network traffic features to distinguish malicious traffic from normal traffic and identify malware types, using common classifiers like logistic regression, K-nearest neighbor (KNN), decision trees, and random forests. Feizollah et al.~\cite{feizollah2013study} also analyzed network traffic using nearly the same set of ML classifiers: KNN, decision trees, Naïve Bayes (NB), multi-layer perceptron (MLP), and support vector machine (SVM). The authors confirmed that KNN delivered the best performance among these classifiers. They also emphasized that new malware constantly emerges, making it necessary to continuously collect samples to train the models and keep them updated.

In summary, much of the related research in the literature focuses on selecting a set of widely used features from network traffic to achieve high accuracy in malware detection using ML algorithms. However, our work differs from these studies in several key aspects.

First, they analyzed a limited number of malware families (a maximum of 18), whereas we considered $116$ malware families. Second, we do not select our features based on the behavior of different malware families, which often requires expert knowledge or referencing previous work. Instead, we allow the machine to learn which features are most relevant for the models from the 3,194 features initially extracted in our experiments. Third, due to the challenge of manually generating input test cases, we used DroidBot~\cite{DBLP:conf/icse/LiYGC17}, an automated test case generation tool, to better simulate user interaction with the apps within a short time frame~\cite{DBLP:conf/kbse/ChoudharyGO15}. DroidBot outperforms similar tools in test case generation, as demonstrated in previous studies~\cite{DBLP:conf/kbse/ChoudharyGO15,DBLP:journals/jss/CostaMMSSBNR22}.


 %Research on deep learning-based malware detection using traffic analysis is limited. Alzaylaee et al.~\cite{ALZAYLAEE2020101663} developed a deep learning model to detect malicious Android apps using static and dynamic features. Wang~\cite{wang:iwqos-2018} proposed a URL-based malware detection method using a multi-view neural network. This network automatically creates multiple views of URLs and assigns attention weights to focus on different features. The method demonstrated high accuracy in detecting malware from different months of a specific year.



\section{Methodology}\label{sec:Methodology}

Our goal is to build an in-depth understanding about the real network traffic generated by apps at the network access point...

\subsection{Malware Dataset}\label{sec:dataset}

\subsection{Traffic Collection Procedure}\label{sec:traffic}

\subsection{Feature Extraction}\label{sec:extraction}

\subsubsection{Flow features set}\label{sec:set}

\subsection{Learning-based Detection Procedures}\label{sec:learning}

\subsection{Understanding Results}\label{sec:understand}
\section{Results}\label{sec:results}

This section presents the results of our research. In Section~\ref{sec:ml}, we compare the
performance of machine learning algorithms in classifying the apps in the \cds dataset
as either malware or non-malware, using network flows collected during the execution of the apps
in conjunction with calls to sensitive APIs. Recall that previous research employed the \emph{vanilla}
\mas, which relied solely on calls to sensitive APIs for app classification. Our previous
research showed that the performance of the vanilla \mas is compromised when consider
the \cds, in particular due to samples from specific malware families {\color{red}(such as ABC)~\cite{}}.
In Section~\ref{sec:new-mas-approach}, we present the gains in classification performance of our extended \mas,
which combines the analysis of sensitive API calls with our designed network flow ML-based classification method.
Finally, in Section~\ref{sec:family-assessment}, we present assessments of our extended
version of the \mas that focus on the malware families responsible for the poor performance of
the vanilla \mas on the \cds.

\subsection{Comparison of Machine Learning Algorithms}\label{sec:ml}


As discussed {\color{red}in Section~\ref{}}, we extended the \mas{} approach to collect network
flow information from the apps during their execution via DroidBot campaigns.
We then replicated our previous experiments using this extended version of the \mas{} approach
that incorporates network flow data collection. Finally, we conducted an experiment to
compare the performance of machine learning algorithms leveraging this network flow data
for malware classification. This step of our research considers the following standard ML algorithms:

\begin{itemize}
 \item Logistic Regression 
 \item Linear Discriminant Analysis (LDA)
 \item Quadratic Discriminant Analysis (QDA)
 \item Random Forest
\end{itemize}

We also explored the Energy-Based Flow Classifier (EFC), which has been used for
intrusion and botnet detection using network flows~\cite{DBLP:journals/tnsm/PontesSGBM21}.
We experimented with multiple parameters and used cross-validation to maximize the performance of
these ML algorithms. In the end, the Random Forest algorithm outperformed the others,
when considering standard metrics (recall, precision, \fone, and Area Under the Curve (AUC)).
Table~\ref{tab:ml-metrics} presents the results. Based on this result,
we explored the research questions using the outputs of the Random Forest
classification.

\begin{table}[htb]
    \caption{Accuracy of the ML algorithms to classify the app as malware or non-malware using network flow data from the \cds.}
  \begin{tabular}{lcccc} \toprule
    Algorithm & Precision & Recall & \fone & AUC \\ \midrule 
    Logistic Regression  & 0.67 & 0.67 & 0.66 & 0.62 \\
    LDA & 0.66 & 0.75 & 0.70 & 0.70 \\
    QDA & 0.63 & 0.68 & 0.65 & 0.68 \\
    EFC & 0.68 & 0.74 & 0.71 & 0.72 \\
    Random Forest & 0.84 & 0.81 & 0.82 & 0.92 \\ \bottomrule    
  \end{tabular}
  \label{tab:ml-metrics}
\end{table}

\begin{finding}
  The Random Forest algorithm outperforms the others popular classification algorithms, achieving higher values across the metrics explored (recall, precision, \fone, and Area Under the Curve).
\end{finding}


% \begin{figure*}[h]
%   \centering  
%     \includegraphics[width=0.85\textwidth]{image/barGraphMetrics.png} \\[\abovecaptionskip]
%   \caption{The comparison of machine learning algorithms}\label{fig:metrics}
% \end{figure*}


\subsection{Comparison of two detection Strategy}\label{sec:new-mas-approach}

{\bf \mas.} Considering the \cds (4,076 apps), the \fhc present
a total of {$1,395$} repackaged apps flagged as malware ({$34.22$\%} of the total number of repackaged apps), for which the repackaged app version calls at least one additional sensitive API. They explore accuracy metrics (such as Precision, Recall, and F-measure ($F_1$)), taking advantage of \vt to label the \cds and building a ground truth. Table~\ref{tab:accuracy} summarize the result of their study (First row). The study indicate that the \mas achieves an accuracy of 0.54 when considering the \cds. Nonetheless, the \mas fails to correctly classify $1,720$ assets as malware on the \cds (FN column, first row of Table~\ref{tab:accuracy}), and wrongly labeled the repackaged version of $220$ apps as malware (FP column). Therefore, the study reveals a {\bf lower performance} related to the accuracy of the approach, indicating that when considering the \cds, the accuracy of the \mas using DroidBot as test generate tool is just over $50$\%.

{\bf Flow Analysis.} Surprisingly, also considering the \cds (\apps pairs), we explored Flow Analysis with machine learning algorithm (Random Forest). As described in Section~\ref{sec:learning}, we trained $70$\% of the samples and tested on $30$\% of samples different from the trained ones. Our \cds cover \apps pair with $2,969$ original apps and $2,918$ malicious apps, totaling $5,887$ balanced samples. Accordingly, we trained our model on $4,120$ samples ($70$\% of $5,887$), and applied the trained model to $1,767$ samples ($30$\% of $5,887$). The Flow Analysis labeled a total of $690$ apps as malware, failed to correctly label $124$ assets as malware, and wrongly labeled the repackaged versions of $175$ samples (second row of Table~\ref{tab:accuracy}). Our Flow Analysis had a better performance, when compared to \mas, with an accuracy rate of $82$\%. From these results, we can conclude that Flow Analysis could be a complementary technique to \mas, improving the identification of malicious code in Android apps. In the next section, we present the results of combining both techniques for suspicious app detection.

\begin{finding}
The experimental results demonstrate that Network Flow analysis, combined with a machine learning algorithm, outperforms the \mas, with \fone of $0.82$. This proves to be an effective strategy to support the \mas for malware identification.
\end{finding}


\begin{table*}[h]
  \caption{Accuracy of both strategy on \cds.}
\centering{
  \begin{tabular}{lrrrrrr} \hline
    Dataset & TP   & FP  & FN  & Precision & Recall & $F_1$ \\
    \hline
    
    %\mas + Traces  & \sds (102)   & 67   & 18  & 2   & 0.78      & 0.97   & 0.87  \\
    \fhc : \mas (4,076 pairs)    & 1,175  & 220 & 1,720 & 0.84       & 0.40   & 0.54  \\
    Flow Analysis (1767 samples)~\footnote{Using Random Forest as ML algorithm}    & 690   & 175   & 124   & 0.79      & 0.84   & 0.82  \\
    %\mas + Traces  & \cds (1203)   & 214  & 326 & 245 & 0.39      & 0.46   & 0.42  \\ 
    \mas and Flow Analysis (4,076 pairs)    & 2,712   & 334   & 183   & 0.89      & 0.93   & 0.91  \\
    \hline
  \end{tabular}
  }
  \label{tab:accuracy}
\end{table*}


\subsection{Combining both Strategy}\label{sec:strategy}

Finally, to confirm our hypothesis from Section~\ref{sec:comparison}, we investigated the benefits of combining both approaches (\mas and Flow Analysis). The combined execution of both techniques correctly classified $2,712$ repackaged apps as malware (TP) and significantly decreased the number of (FN) from $1,720$ to $183$. However, this execution increased the number of (FP) from $220$ to $334$. The combination of both techniques proved to be more effective than the vanilla \mas. In summary, the results reveal that the combination of both techniques achieves an accuracy rate of $91$\% (third row of Table~\ref{tab:accuracy}).

To understand the benefits of each method, we further analyze the contribution of them for the accuracy. We report the raise of True Positive (TP) and False Positive (FP) for each technique in Figure~\ref{fig:venn}. The figure reveal that different approaches present different contribution to the final detection result.

\begin{finding}
When combining both techniques, we improve the overall accuracy (\fone) of \mas at malware detection, from $0.54$ to $0.91$ at \cds.
\end{finding}

\begin{figure}[t!]
  \centering
  \begin{tabular}{@{}c@{}}
    \includegraphics[width=0.54\textwidth]{image/vennTP.png} \\[\abovecaptionskip]
    \small (a) True Positive raise
  \end{tabular}

  \begin{tabular}{@{}c@{}}
    \includegraphics[width=0.54\textwidth]{image/vennFP.png} \\[\abovecaptionskip]
    \small (b) False Positive raise
  \end{tabular}

  \caption{Contribution to the final detection result}\label{fig:venn}
\end{figure}






\subsection{Detection Performance based on Malware Family}\label{sec:family}

In this section, we present the performance of our experiment based on each malware family. Since the number of samples in one family differs from the number in another, the overall detection rate is influenced more by the families with larger sample sizes. However, our results become inconsistent if we use the same number of samples for each family. To resolve the paradox, we present the results of the actual number of samples from the $10$ most representative families, which account for $87.83\%$ of all samples (Section~\ref{sec:familyDetection}). We also explored the detection rate of suspected recent malware samples. Although their families are unknown, we demonstrated that it is possible to improve the detection rate of suspicious apps by combining both strategies (Section~\ref{sec:unknowfamily}).


\subsubsection{Detection rate of 10 most representative malware families}\label{sec:familyDetection}

Among the $10$ most representative families, combining both strategies, the families with the highest earnings are \tjk and \gps. Regarding \tjk family malware, among the 34 samples evaluated, the \fhc flagged only 2 apps ($5.88$\%) as malicious. However, when combined with Flow Analysis, both approaches correctly labeled $32$ assets ($94.11$\%) as malicious, marking an $88.23$\% increase. Despite this, the \tjk family does not have as many representative samples as the \gps family, which accounts for $32.80$\% of all samples in the \cds, with $1,337$ samples. Among them, the \mas flagged $334$ samples ($12.93$\%) as malicious, while the combination of them indicated that $1,275$ ($95.36$\%) had suspicious activity, representing an $82.42$\% increase. Malware belonging to the \gps family automatically connects to networks, communicates with remote servers, and downloads and installs other apps or adware without the user’s knowledge\cite{DBLP:journals/jnca/WangCYYPJ19}. Due to their high network interaction, they are more easily detected by \net, proving it to be more efficient in detecting samples with malicious network behaviors.

Still, we should also note that $2$ families can correctly identified all samples as malware just with the \mas, \fm{airpush} and \fm{leadbolt}, with $120$ and $43$ samples respectively. At this case, the \net do not contribute to improving the ability to detect malicious activity, and just confirms the  maliciousness of the samples. The result reveals that the \mas remain effective for certain malware families. Figure~\ref{fig:bar} shows the detection performance of our strategies for the $10$ most representative families. In the figure, we can see that samples for the \gps family, had the the greatest benefit of the malware detection rate, with the support of flow analysis.



\begin{figure*}[h]
  \centering
  
    \includegraphics[width=\linewidth]{image/barGraph.png} \\[\abovecaptionskip]
    
  \caption{Detection Rate for Family}\label{fig:bar}
\end{figure*}

\begin{finding}

The combination of the \mas with \net, improve the malware detection rate for the \gps family (from $12.93$\% to $95.36$\%), the most representative family in \cds. This demonstrates the potential of this solution in detecting samples with malicious network behaviors.

\end{finding}


\subsubsection{Detection rate of unknown malware family}\label{sec:unknowfamily}

According to \vt, among the samples from our \cds, at least two \ses identified $253$ samples as malware. However, they were unable to specify their families. Since new malware emerges daily, accurately classifying recent malicious apps into their respective families is both challenging and time-consuming~\cite{DBLP:journals/compsec/WangTW21,DBLP:journals/compsec/ContiKP22}, which suggests that these are recent malware.

Although the specific families are unknown, the \mas detected suspicious activity in $114$ samples ($45.05$\%) from this set. On the other hand, when focusing solely on the results from the \net, for these unknown malware families, $124$ samples ($49.01$\%) were flagged as malicious. When we combine both approaches, the total number of apps flagged for suspicious activity increases to $170$ ($67.19$\%). Based on these results, we conclude that \net can effectively support the \mas, even for recently identified malicious apps, without malware family classification at the time of this research.

\begin{finding}

Even for recently malicious apps with unknown malware families, \net can support the \mas to improve the detection rate of apps with suspicious activities.

\end{finding}


\section{Discussion}\label{sec:discussion}

The previous section demonstrated the efficacy of the \droidxpflow framework for detecting malware in network traffic. In this section, we address the research questions posed in Section~\ref{sec:empirical-study}, presenting the implications of our results, and discussing certain limitations that cannot be ignored. These limitations also highlight areas for future research.

\subsection{Research Questions and Analysis}\label{sec:questions}

The assessment of our method in the previous section allows us to answer the research questions as follows:\newline



\begin{enumerate}
    \item \textbf{Accuracy Gain in Android Malware Classification Using Network Flow Data and ML (RQ1).} Our study indicates that the accuracy of \droidxpflow is competitive when compared to the state-of-the-art in malware classification. In our investigation, the framework achieved an \fone of $0.89$ when using the RF algorithm at ML model.
    
    \item \textbf{Machine Learning Algorithms Analysis (RQ2).} Our experimental findings provide evidence that, among all the ML algorithms investigated, the Random Forest algorithm outperformed the other four algorithms tested. It achieved the highest performance according to the relevant metrics for our dataset (\fds).
    
    \item \textbf{Malware Family Detection Accuracy (RQ3).} The results show that certain malware families significantly benefit from the \droidxpflow framework. For example, samples from the \tjk and \gps families achieved correct classification rates above $93\%$. The main characteristics of these malware families include downloading adware without the user’s knowledge, automatically connecting to and interacting with remote servers, and initiating paid services~\cite{DBLP:journals/jnca/WangCYYPJ19}. Therefore, we can confirm that our framework achieves high accuracy, particularly for malware families that frequently engage in malicious network interactions.
\end{enumerate}

\subsection{Implications}\label{sec:implications}

In this section, we highlight some implications based on the results presented in Section~\ref{sec:results}.\newline

Previous studies~\cite{DBLP:conf/wcre/BaoLL18,DBLP:conf/iceccs/LeB0GL18,DBLP:journals/jss/CostaMMSSBNR22} incorrectly identified the \mas as a solution with a reasonable \fone, based on results from a limited dataset composed of fewer than 20 malware families. In contrast, our analysis revealed negative results for the \mas when using a more representative dataset (LargeDS), which included a greater variety of malware families. These families were responsible for higher false negative rates, ultimately compromising the accuracy of the \mas.

Our work addresses this problem, presenting a approach base on network flow analysis with ML support. Our framework proved to be efficient in detect different malicious behaviors and reduce the number of false negatives. More importantly, the framework can identify more malware families that use polymorphism or obfuscation to evade detection~\cite{DBLP:conf/acsac/MoserKK07}, but exhibit high and suspicious interactions with the network. Malware from the \gps and \dwg families are examples of malicious apps that use these strategies.

Still, our study also reveals that for some malware families, \droidxpflow fails to detect their malicious behavior, while the \mas successfully classified them as malware. This proves that \net is not a complete solution, and highlights the importance of combining both approaches. Among all families explored in $30\%$ of \fds 
 (71 families), 11 families have samples that \droidxpflow do not flagged as malicious, however were classified as malicious by the \mas. Examples include the \fm{Dowgin} family (3 samples out of 69) and the \fm{Revmob} family (4 samples out of 67), where only the \mas was able to identify them as malware. Also, the previews chapter show that \mas is able to correctly label as malware, $100\%$ of samples from \fm{Airpush} family. In this specific case, \droidxpflow also classify as malware all samples and just confirms the maliciousness of them. This demonstrates that the current state-of-the-art Mining Sandbox techniques remain effective for certain malware families.

\subsection{Limitations}\label{sec:limitations}

The previous assessment of results, proved that \droidxpflow is an effective approach for malware detection, however it has some limitations that can not be ignored, and are mentioned below:\newline
\textbf{Training set.} The \fds contains $2,886$ malware samples across $116$ families. When considering the number of apps available on official markets today, we realize that our sample is far from representative. Currently, there are millions of apps on Google Play~\cite{bankmycell}, with a significant number of malicious apps hidden among them. We believe that there are still malware families that cannot be detected by our framework. To address this issue, we propose expanding the training set and testing additional detection models. The malware detection capability improves as the size of the training samples increases, enabling the solution to detect more types of malware. Finally, our work also focuses solely on Android repackaged malware, so we cannot generalize our findings to malware targeting other platforms.\newline
\textbf{Malicious behaviors triggered.} As explained in Section~\ref{sec:data}, during the execution phase of DroidXP it restarts the explored apps to presumably activate the malicious behavior of the malware. However, we are uncertain whether all malicious activities were fully triggered without actual user inputs. Bao et al.~\cite{DBLP:conf/wcre/BaoLL18} provides evidence that DroidBot outperforms other test generation tools by uncovering a larger number of potential malicious behaviors. Nevertheless, we are unsure about its ability to accurately simulate user input, which would make the collected traffic resemble real-world scenarios. Furthermore, since we used a simulated environment, it is possible that some malware could detect this situation, and avoid triggering their malicious behaviors, thus affecting the network traffic collection process. In the future, we plan to explore more recent test generation tools that could cover a wider range of app behaviors. Additionally, we intend to incorporate real devices into the traffic collection process to better detect malware that can bypass environment emulators.
\section{Conclusions}\label{sec:conclusions}

In this paper, we propose a method for detecting Android malware by combining the \mas with \net. We use the same dataset (\cds) as the \fhc to investigate whether our method can overcome the limitations of the \mas, point by the \fhc, particularly in families that heavily interact with networks. Our evaluation demonstrates that by combining the \mas with \net, we achieve good performance in detecting mobile malware, with an \fone of $0.91$. Although \net leverage the MAS approach's results, we show that it is not a complete solution. Our evaluation reveals that there are samples in the \cds which \net fault in flag as malicious, but the \mas do not, indicating that both approaches complement each other. Our paper also highlights the limitations posed by the malicious samples, discussing the importance of the number of malware samples used for training, as this can affect the accuracy of machine learning algorithms and, consequently, the support of Network Traffic Analysis for the \mas. As future work, we plan to collect, train, and analyze more malware samples to improve the malware detection solution by developing more sophisticated models, that could provide better support for the \mas.
\balance 

\bibliographystyle{IEEEtran}
\bibliography{ref}


\end{document}


